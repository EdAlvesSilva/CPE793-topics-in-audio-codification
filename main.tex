\documentclass[a4paper]{article} 
\addtolength{\hoffset}{-2.25cm}
\addtolength{\textwidth}{4.5cm}
\addtolength{\voffset}{-3.25cm}
\addtolength{\textheight}{5cm}
\setlength{\parskip}{0pt}
\setlength{\parindent}{0in}

\usepackage[square,sort,comma,numbers]{natbib}
\usepackage{blindtext} % Package to generate dummy text
\usepackage{charter} % Use the Charter font
\usepackage[utf8]{inputenc} % Use UTF-8 encoding
\usepackage{microtype} % Slightly tweak font spacing for aesthetics
\usepackage{amsthm, amsmath, amssymb} % Mathematical typesetting
\usepackage{float} % Improved interface for floating objects
\usepackage{hyperref} % For hyperlinks in the PDF
\usepackage{graphicx, multicol} % Enhanced support for graphics
\usepackage{xcolor} % Driver-independent color extensions
\usepackage{pseudocode} % Environment for specifying algorithms in a natural way
\usepackage[yyyymmdd]{datetime} % Uses YEAR-MONTH-DAY format for dates

\usepackage{fancyhdr} % Headers and footers
\pagestyle{fancy} % All pages have headers and footers
\fancyhead{}\renewcommand{\headrulewidth}{0pt} % Blank out the default header
\fancyfoot[L]{} % Custom footer text
\fancyfoot[C]{} % Custom footer text
\fancyfoot[R]{\thepage} % Custom footer text
\newcommand{\note}[1]{\marginpar{\scriptsize \textcolor{red}{#1}}} % Enables comments in red on margin

%----------------------------------------------------------------------------------------

%\usepackage[maxnames=10]{biblatex}
%-------------------------------
%	TITLE VARIABLES (identify your work!)
%-------------------------------

\usepackage{xcolor}
\usepackage{hyperref}
\usepackage{booktabs}
\usepackage{bbm}

\newcommand{\yourname}{Joao Felipe Guedes da Silva \\ Wallace Abreu} % replace YOURNAME with your name
%\newcommand{\yournetid}{YOURNETID} % replace YOURNETID with your NetID
\newcommand{\youremail}{guedes.joaofelipe@poli.ufrj.br} % replace YOUREMAIL with your email
\newcommand{\assignmenttitle}{}
\newcommand{\assignmentnumber}{1}

\begin{document}

%-------------------------------
%	TITLE SECTION (do not modify unless you really need to)
%-------------------------------
\fancyhead[C]{}
\hrule \medskip
\begin{minipage}{0.295\textwidth} 
\raggedright
\footnotesize
\yourname \hfill\\ 
\yournetid \hfill\\ 
\youremail
\end{minipage}
\begin{minipage}{0.4\textwidth} 
\centering 
\large 
CPE793 - Tópicos em Codificação de Áudio - 2021.2\\ 
\normalsize 
Lista \assignmentnumber\\ 
\end{minipage}
\begin{minipage}{0.295\textwidth} 
\raggedleft
\today\hfill\\
\end{minipage}
\medskip\hrule 
\bigskip


%-------------------------------
%	ASSIGNMENT CONTENT (add your responses)
%-------------------------------
\section{\textit{Filhos e Filhas}: Considere um casal que tem dois descendentes e que as chances de cada um deles ser filho ou filha são iguais. Responda às perguntas abaixo:}

\subsection{Calcule a probabilidade dos descendentes formarem um casal (ou seja, um filho e uma filha)}.

Seja A o elemento ``descendente ser filha" e O o elemento ``descendente ser filho. Assim, como as chances de cada um deles ser filho ou filha são iguais: 

$$P_A = P_O = 1/2$$

Considerando que o casal terá dois descendentes, o espaço amostral de possíveis pares de descendentes é: 

$$S = [\{A, O\}, \{A, A\}, \{O, O\}, \{O, A\}]$$

Os eventos de interesse, portanto, são: $\{A, O\}$ ou $\{O, A\}$.

A probabilidade de cada um desses eventos é:

$$P[\{A, O\}] = P[A \wedge	 O] = 1/2 \cdot 1/2 = 1/4$$
$$P[\{O, A\}] = P[O \wedge	 A] = 1/2 \cdot 1/2 = 1/4$$

Logo, a probabilidade de se ter um casal é:

$$P_{casal} = P[\{A, O\} \vee \{O, A\}] = 1/4 + 1/4 = 1/2$$

\subsection{Calcule a probabilidade de ao menos um dos descendentes ser filho.}

Neste caso, os eventos de interesse são: $\{A, O\}, \{O, O\}, \{O, A\}]$.

Portanto, 

$$P_{1\ filho} = P[\{A, O\} \vee \{O, O\} \vee \{O, A\}] = 1/4 + 1/4 + 1/4 = 3/4$$


\subsection{Calcule a probabilidade das duas serem filhas dado que uma é filha (cuidado!)}

Dado que 1 é filha, o espaço amostral é reduzido para o sub-conjunto: 

$$S_A = [\{A, O\}, \{A, A\}, \{O, A\}]$$.

\begin{equation}
    \begin{split}
        P[\{A, A\} | S_A] &= \frac{P[\{A, A\} \cap S_A]}{P[S_A]} \\
            &= \frac{P[\{A, A\}]}{P[S_A]} \\
            &= \frac{1/4}{3/4} \\
            &= 1/3
    \end{split}
\end{equation}

\subsection{Calcule a probabilidade dos descendentes nascerem no mesmo dia (assuma que a chance de nascer em um determinado dia é igual a qualquer outro).}

Neste caso, podemos ignorar o sexo do descendente e analisar apenas o dia em que ele/ela nasceu. \\

Considerando que o ano possui 365 dias e a probabilidade do descendente nascer no dia $i$ é $P_i = 1/365, \forall i$, então: 

$$P[D_1 = i \wedge D_2 = i]$$

Onde $D_1$ e $D_2$ são as datas de nascimento do primeiro e segundo descendente, respectivamente. Considerando que o nascimento de ambos sejam eventos independentes: 

$$P[D_1 = i \wedge D_2 = i] = P[D_1 = i] \cdot P[D_2 = i] = \Big( 1/365 \Big)^2$$





%\bibliographystyle{acm}
%\bibliography{references} % citation records are in the references.bib document

\end{document}
