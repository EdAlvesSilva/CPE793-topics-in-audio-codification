\documentclass[a4paper]{article} 
\addtolength{\hoffset}{-2.25cm}
\addtolength{\textwidth}{4.5cm}
\addtolength{\voffset}{-3.25cm}
\addtolength{\textheight}{5cm}
\setlength{\parskip}{0pt}
\setlength{\parindent}{0in}

\usepackage[square,sort,comma,numbers]{natbib}
\usepackage{blindtext} % Package to generate dummy text
\usepackage{charter} % Use the Charter font
\usepackage[utf8]{inputenc} % Use UTF-8 encoding
\usepackage{microtype} % Slightly tweak font spacing for aesthetics
\usepackage{amsthm, amsmath, amssymb} % Mathematical typesetting
\usepackage{float} % Improved interface for floating objects
\usepackage{hyperref} % For hyperlinks in the PDF
\usepackage{graphicx, multicol} % Enhanced support for graphics
\usepackage{xcolor} % Driver-independent color extensions
\usepackage{pseudocode} % Environment for specifying algorithms in a natural way
\usepackage[yyyymmdd]{datetime} % Uses YEAR-MONTH-DAY format for dates

\usepackage{fancyhdr} % Headers and footers
\pagestyle{fancy} % All pages have headers and footers
\fancyhead{}\renewcommand{\headrulewidth}{0pt} % Blank out the default header
\fancyfoot[L]{} % Custom footer text
\fancyfoot[C]{} % Custom footer text
\fancyfoot[R]{\thepage} % Custom footer text
\newcommand{\note}[1]{\marginpar{\scriptsize \textcolor{red}{#1}}} % Enables comments in red on margin

%----------------------------------------------------------------------------------------

%\usepackage[maxnames=10]{biblatex}
%-------------------------------
%	TITLE VARIABLES (identify your work!)
%-------------------------------

\usepackage{xcolor}
\usepackage{hyperref}
\usepackage{booktabs}
\usepackage{bbm}

\newcommand{\yourname}{Joao Felipe Guedes da Silva} % replace YOURNAME with your name
%\newcommand{\yournetid}{YOURNETID} % replace YOURNETID with your NetID
\newcommand{\youremail}{guedes.joaofelipe@poli.ufrj.br} % replace YOUREMAIL with your email
\newcommand{\assignmenttitle}{}
\newcommand{\assignmentnumber}{1}

\begin{document}

%-------------------------------
%	TITLE SECTION (do not modify unless you really need to)
%-------------------------------
\fancyhead[C]{}
\hrule \medskip
\begin{minipage}{0.295\textwidth} 
\raggedright
\footnotesize
\yourname \hfill\\ 
\yournetid \hfill\\ 
\youremail
\end{minipage}
\begin{minipage}{0.4\textwidth} 
\centering 
\large 
CPE793 - Tópicos em Codificação de Áudio - 2021.2\\ 
\normalsize 
Lista \assignmentnumber\\ 
\end{minipage}
\begin{minipage}{0.295\textwidth} 
\raggedleft
\today\hfill\\
\end{minipage}
\medskip\hrule 
\bigskip


%-------------------------------
%	ASSIGNMENT CONTENT (add your responses)
%-------------------------------
\section{\textit{Filhos e Filhas}: Considere um casal que tem dois descendentes e que as chances de cada um deles ser filho ou filha são iguais. Responda às perguntas abaixo:}

\subsection{Calcule a probabilidade dos descendentes formarem um casal (ou seja, um filho e uma filha)}.

Seja A o elemento ``descendente ser filha" e O o elemento ``descendente ser filho. Assim, como as chances de cada um deles ser filho ou filha são iguais: 

$$P_A = P_O = 1/2$$

Considerando que o casal terá dois descendentes, o espaço amostral de possíveis pares de descendentes é: 

$$S = [\{A, O\}, \{A, A\}, \{O, O\}, \{O, A\}]$$

Os eventos de interesse, portanto, são: $\{A, O\}$ ou $\{O, A\}$.

A probabilidade de cada um desses eventos é:

$$P[\{A, O\}] = P[A \wedge	 O] = 1/2 \cdot 1/2 = 1/4$$
$$P[\{O, A\}] = P[O \wedge	 A] = 1/2 \cdot 1/2 = 1/4$$

Logo, a probabilidade de se ter um casal é:

$$P_{casal} = P[\{A, O\} \vee \{O, A\}] = 1/4 + 1/4 = 1/2$$

\subsection{Calcule a probabilidade de ao menos um dos descendentes ser filho.}

Neste caso, os eventos de interesse são: $\{A, O\}, \{O, O\}, \{O, A\}]$.

Portanto, 

$$P_{1\ filho} = P[\{A, O\} \vee \{O, O\} \vee \{O, A\}] = 1/4 + 1/4 + 1/4 = 3/4$$


\subsection{Calcule a probabilidade das duas serem filhas dado que uma é filha (cuidado!)}

Dado que 1 é filha, o espaço amostral é reduzido para o sub-conjunto: 

$$S_A = [\{A, O\}, \{A, A\}, \{O, A\}]$$.

\begin{equation}
    \begin{split}
        P[\{A, A\} | S_A] &= \frac{P[\{A, A\} \cap S_A]}{P[S_A]} \\
            &= \frac{P[\{A, A\}]}{P[S_A]} \\
            &= \frac{1/4}{3/4} \\
            &= 1/3
    \end{split}
\end{equation}

\subsection{Calcule a probabilidade dos descendentes nascerem no mesmo dia (assuma que a chance de nascer em um determinado dia é igual a qualquer outro).}

Neste caso, podemos ignorar o sexo do descendente e analisar apenas o dia em que ele/ela nasceu. \\

Considerando que o ano possui 365 dias e a probabilidade do descendente nascer no dia $i$ é $P_i = 1/365, \forall i$, então: 

$$P[D_1 = i \wedge D_2 = i]$$

Onde $D_1$ e $D_2$ são as datas de nascimento do primeiro e segundo descendente, respectivamente. Considerando que o nascimento de ambos sejam eventos independentes: 

$$P[D_1 = i \wedge D_2 = i] = P[D_1 = i] \cdot P[D_2 = i] = \Big( 1/365 \Big)^2$$


\section{\textit{Dado:} Considere um icosaedro (um sólido Platônico de 20 faces) tal que a chance de sair a face $i = 1, \ldots, 20$ seja linearmente proporcional a $i$. Ou seja, $P[X = i] = ci$ para alguma constante $c$, onde $X$ é uma variável aleatória que denota a face do dado. Responda às perguntas abaixo:}

\subsection{Determine o valor de $c$.}

$$f_X(i) = P[X = i] = ci$$

Sabemos que uma variável aleatória possui a seguinte restrição: 

$$\sum_{i = i}^{20} f_X(i) = 1$$

Consequentemente, 

\begin{equation*}
    \begin{split}
        \sum_{i = i}^{20} f_X(i) &= \sum_{i = 1}^{20} ci \\
            &= c\cdot(1+2+\ldots+20) \\
            &= c \cdot \Big[ \frac{20\cdot (1+20)}{2} \Big] \\
            &= c \cdot 210 \\
            &= 1
    \end{split}
\end{equation*}

Logo, $c = 1/210$.

\subsection{Calcule o valor esperado de $X$ (obtenha também o valor numérico).}

\begin{equation*}
    \begin{split}
        E[X] &= \sum_{i = 1}^{20} i\cdot f_X(i)\\
            &= \frac{1}{210} \sum_{i = i}^{20} i\cdot i\\
            &= \frac{1}{210} (1 + 2^2 + \ldots + 20^2) \\
            &= 13.67
    \end{split}
\end{equation*}

\subsection{Calcule a probabilidade de X ser maior do que seu valor esperado.}

os eventos de interesse que fazem com que $X > E[X]$ são: $\{14, 15, \ldots, 20\}$. Assim, a probabilidade de que isso aconteça é: 

$$P[X > E[X]] = P[\{14, 15, \ldots, 20\}] = \sum_{i=14}^{20} \frac{i}{210} = 57\%$$


\subsection{Calcule a variância de X (obtenha também o valor numérico).}

Pela definição, 

\begin{equation*}
    \begin{split}
        \sigma_X^2 &= E[(X - E[X])^2]\\
            &= E[X^2 - 2X\cdot E[X] + E^2[X]] \\
            &= E[X^2] - E^2[X]
    \end{split}
\end{equation*}

Considerando $g(X) = X^2$, temos:

\begin{equation*}
    \begin{split}
        E[X^2] &= E[g(X)] \\
            &= \sum_{i = 1}^{20} g(i)\cdot f_X(i) \\
            &= \frac{1}{210}\sum_{i = 1}^{20} i^2\cdot i \\
            &= \frac{1}{210} (1 + 2^3 + \ldots + 20^3) \\
            &= 210
    \end{split}
\end{equation*}

Portanto, 

\begin{equation*}
    \begin{split}
        \sigma_X^2 &= E[X^2] - E^2[X]\\
            &= 210 - 13.67^2 \\
            &= 23.22
    \end{split}
\end{equation*}


\subsection{Repita os últimos três itens para o caso do dado ser uniforme, ou seja, $P[X = i] = 1/20, i = 1, \ldots, 20$. Qual dado possui maior variância? Compare os resultados.}

\subsubsection{Determine o valor de c.}

Neste caso, $$f_X(i) = P[X = i] = c\cdot i = 1/20$$. Portanto, c = $i/20$.

\subsubsection{Calcule o valor esperado de $X$ (obtenha também o valor numérico).}

Aplicando a definição, temos: 

\begin{equation*}
    \begin{split}
        E[X] &= \sum_{i = 1}^{20} i\cdot f_X(i) \\
            &= \sum_{i = 1}^{20} i\cdot \frac{1}{20} \\
            &= 10.5
    \end{split}
\end{equation*}

\subsubsection{Calcule a probabilidade de X ser maior do que seu valor esperado.}

Os eventos de interesse que fazer com que $X > E[X]$ são: $\{11, 12, \ldots, 20\}$. Desta forma, a probabilidade de X ser maior do que seu valor esperado é: 

$$P[X > E[X]] = P[\{11, 12, \ldots, 20\}] = \sum_{i=11}^{20}\frac{1}{20} = \frac{9}{20} = 45\%$$


\subsubsection{Calcule a variância de X (obtenha também o valor numérico).}

Calculando $E[X^2]$: 

\begin{equation*}
    \begin{split}
        E[X^2] &= \sum_{i = 1}^{20} i^2 \cdot f_X(i) \\
            &= \sum_{i = 1}^{20} i^2\cdot \frac{1}{20} \\
            &= \frac{1}{20}(1 + 2^2 + \ldots + 20^2) \\
            &= 143.5
    \end{split}
\end{equation*}

Portanto, o valor da variância é: 

\begin{equation*}
    \begin{split}
        \sigma_X^2 &= E[X^2] - E^2[X]\\
            &= 143.5 - 10.5^2 \\
            &= 33.25 
    \end{split}
\end{equation*}

Como podemos ver, o dado com fazer de probabilidade uniforme possui uma variância \textbf{maior}. Isso ocorre pois, no primeiro caso, onde a probabilidade da face é proporcional ao seu valor, as chances de sairem faces com valor alto são maiores. Desta forma, a maioria das jogadas terá faces com valores altos e, consequentemente, a variância será menor e no entorno de uma média mais alta. 


\section{\textit{Dado em ação}Considere a versão uniforme do dado acima, ou seja, $P[X = i] = 1=20, i = 1, \ldots, 20$. Seja $Y$ uma variável aleatória indicadora da primalidade da face do dado. Ou seja, $Y = 1$ quando o $X$ é um número primo, e $Y = 0$ caso contrário. Responda às perguntas abaixo:}

\subsection{Determine $Y[P = 1]$.}

$Y[P = 1]$ ocorre quando qualquer um dos elementos abaixo ocorre: 

$$\{2, 3, 5, 7, 11, 13, 17, 19\}$$

Como todos os elementos possuem igual probabilidade, $Y[P = 1] = 8/20 = 2/5$.

\subsection{Considere que o dado será jogado $n$ vezes. Seja $Y_i$ a indicadora da primalidade da i-ésima rodada, para $i = 1, \ldots, n$, e defina $Z = \sum_{i=1}^n Y_1$. Repare que $Z$ é uma variável aleatória que denota o número de vezes que o resultado é primo. Determine a distribuição de $Z$, ou seja, $P[Z = k]$, para $k = 0, \ldots, n$. Que distribuição é esta?}

Considerando que a variável aleatória Y pode ser modelada como uma Bernoulli com probabilidade $Y[P = 1] = 2/5$, então a variável $Z$ será uma soma de várias Bernoulli: 

$$Z = \sum_{i = 1}^n Bernoulli(2/5)$$

Consequentemente, a variável $Z$ terá uma distribuição Binomial com parâmetros $n$ e $p = 2/5$.

$$Z = Binom(n, 2/5)$$

\subsection{Considere que o dado será jogado até que um número primo seja obtido. Seja $Y_i$ a indicadora da primalidade da i-ésima rodada, para $i = 1, \ldots$, e defina $Z = min_{\{i|Y_i = 1\}}$. Repare que $Z$ denota o número de vezes que o dado é jogado até que o resultado seja um número primo. Determine a distribuição de $Z$, ou seja $P[Z = k]$, para $k = 1, \ldots$. Que distribuição é esta?}

Neste caso, a variável $Y$ continua sendo modelada como uma Bernoulli de parâmetro $2/5$:

$$Y = Bernoulli(2/5)$$

No entanto, a variável Z terá uma distribuição geométrica com parâmetro $2/5$:

$$Z = Geom(2/5)$$


\section{\textit{Cobra:} Considere três imagens tiradas em uma floresta, $I_1$, $I_2$, e $I_3$. Em apenas uma das imagens existe uma pequena cobra. Um algoritmo de detecão de cobras em imagens detecta a cobra na imagem $i$ com probabilidade $\alpha_i$. Suponha que o algoritmo não encontrou a cobra na imagem $I_1$. Defina o espaço amostral e os eventos apropriados e use regra de Bayes para determinar:}

\subsection{A probabilidade da cobra estar na imagem $I_1$.}

Modelaremos este problema da seguinte forma: os elementos do espaço amostral são imagens que, quando obtidas em sequência, formam eventos. Esses eventos, portanto, são tríados no espaço amostral. \\

Uma variável aleatória $E$ neste espaço amostral mapeia cada evento no conjunto de números inteiros $\{1, 2, 3\}$. Assim, denotaremos $E_i$ como o algoritmo detectar a cobra no elemento $I_i$ e mapear em $i$.\\

Uma segunda variável aleatória diz respeito à existência (ou não) da propriedade ``a imagem possui uma cobra".  Assim, quando existe uma cobra na imagem $i$, denotamos $C_i = 1$. Caso contrário, $C_i$ = 0. \\

Portanto, segue as definições: 

\begin{itemize}
    \item $C_i$: cobra está na imagem $I_i$
    \item $E_i$: algoritmo detecta a cobra na imagem $I_i$
\end{itemize}

onde $i \in \{1,2,3\}$. Do enunciado, temos as seguintes probabilidades condicionais:

$$P[E_i|C_i] = \alpha_i, \forall i \in \{1,2,3\}$$ 

Consequentemente, 

$$P[\neg E_i|C_i] = (1-\alpha_i), \forall i \in \{1,2,3\}$$ 

Consideraremos que a probabilidade da cobra estar em qualquer imagem é uniforme, ou seja: 

$$P[C_i] = 1/3, \forall i \in \{1,2,3\}$$

E, além disso, consideraremos que não há falsos positivos, ou seja: 

$$P[C_i|E_i] = 1, \forall i \in \{1,2,3\}$$

Queremos descobrir $P[C_1|\neg E_1]$, ou seja, a probabilidade de termos um falso negativo. Pela regra de Bayes, temos que: 

$$P[C_1|\neg E_1] = \frac{P[\neg E_1|C_1]P[C_1]}{P[E_1]} = \frac{(1-\alpha_1)\cdot 1/3}{P[\neg E_1]} = \frac{(1-\alpha_1)\cdot 1/3}{1- P[E_1]}$$

Ainda pela regra de Bayes: 

$$P[E_1] = \frac{P[E_1|C_1]\cdot P[C_1]}{P[C_1|E_1]} = \frac{\alpha_1 \cdot 1/3}{1}$$

Portanto: 

$$P[C_1|\neg E_1] = \frac{1-\alpha_1}{3-\alpha_1}$$

%Pessoal, na 4, tentem encontrar P[E_i] utilizando a regra de Bayes, e após isso tentem utilizar a Lei da Probabilidade Total para "expandir" P[C_i] considerando os eventos E_i e not E_i


\subsection{A probabilidade da cobra estar na imagem $I_2$.}

Neste caso, queremos descobrir $P[C_2|\neg E_1]$:

$$P[C_2|\neg E_1] = \frac{P[\neg E_1|C_2]P[C_1]}{P[E_1]} = \frac{\Big(1-P[E_1|C_2]\Big)P[C_1]}{P[E_1]}$$

Pela regra da probabilidade total, $P[E_1] = \sum_{j=1}^3 P[E_1|C_j]P[C_j] = 1/3\cdot \sum_{j=1}^3 P[E_1|C_j]$\\


Pela regra da probabilidade total, $P[C_i] = \sum_{j=1}^3 P[C_i|E_j]P[E_j] = 1/3$


\section{\textit{Sem memória: } Seja $X \approx Geo(p)$ uma variável aleatória Geométrica com parâmetro $p$. Mostre que a distribuição geométrica não tem memória. Ou seja, dado que $X > k$, o número de rodadas adicionais até que o evento de interesse ocorra possui a mesma distribuição (dica: formalize esta afirmação).}\\

Para este problema, seria útil termos a CCDF da distribuição exponencial, que nos dá a probabilidade $P[X > k]$. Assim, iniciaremos calculando a CDF desta distribuição: 

\begin{equation*}
    \begin{split}
        P[X < k] &= \sum_{i = 1}^k P[X = i] \\
            &= \sum_{i = 1}^k p(1-p)^{i-1}\\
            &= p\sum_{i = 1}^k (1-p)^{i-1}\\
            &= p\frac{1\cdot[(1-p)^{k-1}-1]}{1-p-1}\\
            &= 1-(1-p)^{k-1}
    \end{split}
\end{equation*}

Consequentemente, a CCDF desta distribuição é dada por: 

$$P[X > k] = 1 - P[X \leq k] = (1-p)^{k-1}$$

Por outro lado, queremos analisar o número de rodadas $s$ adicionais dado que um evento ocorreu na rodada $k$. Assim, utilizando a definição da probabilidade condicional, temos:

$$P[X > s + k | X > k] = \frac{P[X > s + k \cap X > k]}{P[X > k]}$$

Intuitivamente, vemos que a interseção $P[X > s + k \cap X > k]$ para s > 0 é $P[X > s + k]$. Logo, 

\begin{equation*}
    \begin{split}
        P[X > s + k | X > k] &= \frac{P[X > s + k]}{P[X > k]}\\
            &= \frac{(1-p)^{s+k-1}}{(1-p)^{k-1}}\\
            &= (1-p)^{s-1} \\
            &= P[X > s]
    \end{split}
\end{equation*}

Portanto, vemos que a distribuição do número de rodadas $s$ adicionais segue a mesma distribuição. 


\section{Considere que o processo de chegada do ônibus 485 no ponto do CT seja bem representado por um processo de Poisson. Ou seja, $X \approx Poi(\lambda; t)$ denota o número (aleatório) de ônibus que chegam ao ponto em um intervalo de tempo $t$ com taxa média de chegada igual à $\lambda$. Assuma que $\lambda = 10$ ônibus por hora.}


\subsection{Determine a probabilidade de não chegar nenhum ônibus em um intervalo de 30 minutos.}

A pdf de uma variável aleatória que segue uma distribuição de Poisson possui a seguinte formulação: 

$$f(k, \lambda) = P[X = k] = \frac{\lambda^ke^{-\lambda}}{k!}$$

Assim, ela nos indica a probabilidade de $k$ eventos ocorrerem em um intervalo $t$ dado que os eventos ocorrem a uma taxa $\lambda$ (medido em $t$). \\

Como o processo de chegada do 485 acontece a uma taxa $\lambda = 10$ ônibus por hora, podemos considerar que em 30 minutos ele ocorre a uma taxa de $\lambda_1 = 5$ ônibus a cada meia hora. Desta forma,  para determinar a probabilidade de não chegar nenhum ônibus em um intervalo de 30 minutos, fazemos: 

$$P[X = 0\ para\ \lambda_1] = \frac{\lambda_1^0e^{-5}}{0!} = e^{-5}$$

\subsection{Determine a probabilidade de 3 ônibus chegarem dentro de um intervalo de 5 minutos.}

Analogamente, temos que $\lambda_2 = 10\cdot 5 / 60 = 5/6$ ônibus a cada 5 minutos. A probabilidade de que 3 ônibus cheguem neste intervalo é: 

$$P[X = 3\ para\ \lambda_2] = \frac{\lambda_2^3e^{-\lambda_2}}{3!} =4.19\%$$


\subsection{Determine a probabilidade da média ocorrer, ou seja, de chegarem exatamente 10 ônibus em uma hora.}

$$P[X = 10\ para\ \lambda] = \frac{\lambda^{10}e^{-\lambda}}{10!} = 12.5\%$$


\section{\textit{Propriedades:} Seja $X$ e $Y$ duas variáveis aleatórias discretas. Mostre as seguintes equivalências usando as definições:}

\subsection{$E[X] = E[E[X|Y]]$, conhecida como regra da torre da esperança}

Seja $O_X$ e $O_Y$ os espaços de valoress que as variáveis $X$ e $Y$ podem ter, respectivamente. Assim, a probabilidade condicional pode ser definida da seguinte forma: 

$$P[X|Y] = \frac{P[X = x; Y = y]}{P[Y = y]}$$

Além disso, pela definição temos a seguinte formulação de valor esperado: 

$$E[X| Y = y] = \sum x \cdot P[X|Y = y], \forall y \in O_Y$$

Aqui, considera-se que Y é uma variável aleatória que faz com que o valor esperado $E[X| Y = y]$ também seja uma variável aleatória. Desta forma, pode-se aplicar o valor esperado desta expressão: 

\begin{equation*}
    \begin{split}
        E[E[X| Y = y]] &= \sum_{x \in O_X} E[X| Y = y]P[Y = y] \\
            &= \sum_{x \in O_X} \sum_{y \in O_Y} x \cdot P[X|Y = y]P[Y = y]\\
            &= \sum_{x \in O_X} \sum_{y \in O_Y} x \cdot \frac{P[X = x; Y = y]}{P[Y = y]}P[Y = y] \\
            &= \sum_{x \in O_X} \sum_{y \in O_Y} x \cdot P[X = x; Y = y] \\
            &= \sum_{x \in O_X} x \cdot P[X = x] \\
            &= E[X]
    \end{split}
\end{equation*}

\subsection{$Var[X] = E[X^2]-E[X]^2$}

\begin{equation*}
    \begin{split}
        Var[X] &= E[(X-E[X])^2] \\
            &= E[X^2 - 2X\cdot E[X] + E[X]^2] \\
            &= E[X^2] -2E[X]E[X] + E[X]^2 \\
            &= E[X^2] - E[X]^2
    \end{split}
\end{equation*}



\section{\textit{Caras em sequência: } Considere uma moeda enviesada, tal que o probabilidade do resultado ser cara é p (e coroa $1-p$). Considere o número de vezes que a moeda precisa ser jogada para obtermos $k$ caras consecutivas. Por exemplo, na sequência $COCOCCOOCOCCC$ a moeda teve que ser jogada $13$ vezes até o aparecimento de $k = 3$ caras consecutivas, onde $C = cara$ e $O = coroa$. Seja $N_k$ a variável aleatória que denota esta quantidade. Qual é o número médio de vezes que a moeda precisa ser jogada para obtermos $k$ caras consecutivas, ou seja, qual é o valor esperado de $N_k$? Dica: monte uma recursão e use a regra da torre da esperança.}\\

Considere $Y$ uma variável indicadora auxiliar que indica a $1ª$ ocorrência de uma coroa. Neste caso, $Y$ segue uma distribuição geométrica com parâmetro $1-p$ que pode assumir valores de $0$ a $\infty$.\\

Caso $Y \geq k + 1$, então $k$ caras seguidas ocorreram. Caso contrário, ao menos 1 coroa aconteceu no índice $Y = y$ e a contagem de caras consecutivas é reiniciada. \\

Isto pode ser formulado matematicamente da seguinte forma: 


\[
    E[N_k|Y] = \left\{
                \begin{array}{ll}
                  y+E[N_k], y \leq k\\
                  k, y \geq k + 1
                \end{array}
              \right.
  \]
  
  O valor de $E[N_k]$, portanto, pode ser obtido aplicando-se a propriedade da torre da esperança: 
  
  \begin{equation*}
    \begin{split}
        E[N_k] &= E[E[N_k|Y]]  \\
            &= \sum_{y = 1}^\infty E[N_k|Y = y]P[Y = y]\\
            &= \sum_{y = 1}^k(y + E[N_k])P[Y = y] + \sum_{y = k+1}^\infty kP[Y = y]
    \end{split}
  \end{equation*}
  
  Uma vez que Y segue uma distribuição geométrica: $P[Y = y] = (1-(1-p))^{y-1} = p^{y-1}$
  
  Assim, 
  
  \begin{equation*}
      \begin{split}
          E[N_k] &= \sum_{y = 1}^k(y + E[N_k])p^{y-1} + \sum_{y = k+1}^\infty kp^{y-1} \\
            &= \sum_{y = 1}^kyp^{y-1} + E[N_k]\sum_{y = 1}^kp^{y-1} + k\sum_{y = k+1}^\infty p^{y-1}
      \end{split}
  \end{equation*}

O terceiro termo da soma é uma PG infinita que começa em $p^k$ com razão $p$. Assim, 

$$k\sum_{y = k+1}^\infty p^{y-1} = k\frac{p^k}{1-p}$$

O segundo termo da soma é uma PG finita com razão $p$ que vai de $1$ a $k-1$:

$$E[N_k]\sum_{y = 1}^kp^{y-1} = E[N_k] \Big( \frac{1 \cdot (p^k-1)}{p-1}\Big)$$

O primeito termo da soma é uma PAG finita com razão da PG $p$ e razão da PA $1$. Chamaremos este termo de $PAG$.

Portanto, 

$$E[N_k] = \frac{PAG+k\frac{p^k}{1-p}}{1-\frac{p^k-1}{p-1}}$$



%oi Vinícius, então, nessa questão a gente tá interessado no valor de esperado de N_k, que é um número, e q podemos definir em função de E[N_k | Y]. N_k em si é uma v.a. que já foi definida, é o número de rodadas que precisamos jogar até termos caras consecutivas. Essa é a definição, nós estabelecemos o que aquela v.a. representa. Quando vc escreve N_k = Y + N_k, você está dizendo que o valor de uma realização da v.a. N_k é igual a uma realização de Y mais uma outra realização de N_k. Essas duas realizações são as mesmas ou diferentes? Isso vale sempre, para todo valor que N_k pode assumir? As respostas dessas perguntas não estão claras

%a v.a. N_k já foi estabelecida, o q a gente quer agora é falar alguma coisa sobre o valor esperado de N_K | Y

%Manipular v.a.'s de maneira recursiva é uma parada que tem q ser feita com cuidado, como no exemplo que o Daniel deu na aula sobre "X + X = 2X"

%aliás, deixa eu me corrigir rapidinho: a gente não tá definindo E[N_k] de maneira recursiva, mas estamos sim fazendo E[N_k | Y] em função de E[N_k]. Não entra recursão nessa parte. E vale enfatizar mais uma vez que E[N_k | Y] e E[N_k] são números, que podem ser manipulados em equações


%\bibliographystyle{acm}
%\bibliography{references} % citation records are in the references.bib document

\end{document}
